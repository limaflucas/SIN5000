
%% bare_jrnl_compsoc.tex
%% V1.4b
%% 2015/08/26
%% by Michael Shell
%% See:
%% http://www.michaelshell.org/
%% for current contact information.
%%
%% This is a skeleton file demonstrating the use of IEEEtran.cls
%% (requires IEEEtran.cls version 1.8b or later) with an IEEE
%% Computer Society journal paper.
%%
%% Support sites:
%% http://www.michaelshell.org/tex/ieeetran/
%% http://www.ctan.org/pkg/ieeetran
%% and
%% http://www.ieee.org/

%%*************************************************************************
%% Legal Notice:
%% This code is offered as-is without any warranty either expressed or
%% implied; without even the implied warranty of MERCHANTABILITY or
%% FITNESS FOR A PARTICULAR PURPOSE! 
%% User assumes all risk.
%% In no event shall the IEEE or any contributor to this code be liable for
%% any damages or losses, including, but not limited to, incidental,
%% consequential, or any other damages, resulting from the use or misuse
%% of any information contained here.
%%
%% All comments are the opinions of their respective authors and are not
%% necessarily endorsed by the IEEE.
%%
%% This work is distributed under the LaTeX Project Public License (LPPL)
%% ( http://www.latex-project.org/ ) version 1.3, and may be freely used,
%% distributed and modified. A copy of the LPPL, version 1.3, is included
%% in the base LaTeX documentation of all distributions of LaTeX released
%% 2003/12/01 or later.
%% Retain all contribution notices and credits.
%% ** Modified files should be clearly indicated as such, including  **
%% ** renaming them and changing author support contact information. **
%%*************************************************************************


% *** Authors should verify (and, if needed, correct) their LaTeX system  ***
% *** with the testflow diagnostic prior to trusting their LaTeX platform ***
% *** with production work. The IEEE's font choices and paper sizes can   ***
% *** trigger bugs that do not appear when using other class files.       ***                          ***
% The testflow support page is at:
% http://www.michaelshell.org/tex/testflow/


\documentclass[10pt,journal,compsoc]{IEEEtran}
%
% If IEEEtran.cls has not been installed into the LaTeX system files,
% manually specify the path to it like:
% \documentclass[10pt,journal,compsoc]{../sty/IEEEtran}


% % Personal packages
% Support for accentuation
\usepackage[utf8]{inputenc}
\usepackage[brazilian]{babel}
% Adding pdf bookmark
\usepackage{bookmark}
% Inline lists
\usepackage[inline]{enumitem}
% Fancy way to display text
\usepackage[ruled, vlined]{algorithm2e}
% Page change
\usepackage[strict]{changepage}
% Landscape orientation
\usepackage{lscape}
% Multipage table
\usepackage{longtable}
% Table cells adjustment
\usepackage{tabularx}
% Some very useful LaTeX packages include:
% (uncomment the ones you want to load)


% *** MISC UTILITY PACKAGES ***
%
%\usepackage{ifpdf}
% Heiko Oberdiek's ifpdf.sty is very useful if you need conditional
% compilation based on whether the output is pdf or dvi.
% usage:
% \ifpdf
%   % pdf code
% \else
%   % dvi code
% \fi
% The latest version of ifpdf.sty can be obtained from:
% http://www.ctan.org/pkg/ifpdf
% Also, note that IEEEtran.cls V1.7 and later provides a builtin
% \ifCLASSINFOpdf conditional that works the same way.
% When switching from latex to pdflatex and vice-versa, the compiler may
% have to be run twice to clear warning/error messages.






% *** CITATION PACKAGES ***
%
\ifCLASSOPTIONcompsoc
  % IEEE Computer Society needs nocompress option
  % requires cite.sty v4.0 or later (November 2003)
  \usepackage[nocompress]{cite}
\else
  % normal IEEE
  \usepackage{cite}
\fi
% cite.sty was written by Donald Arseneau
% V1.6 and later of IEEEtran pre-defines the format of the cite.sty package
% \cite{} output to follow that of the IEEE. Loading the cite package will
% result in citation numbers being automatically sorted and properly
% "compressed/ranged". e.g., [1], [9], [2], [7], [5], [6] without using
% cite.sty will become [1], [2], [5]--[7], [9] using cite.sty. cite.sty's
% \cite will automatically add leading space, if needed. Use cite.sty's
% noadjust option (cite.sty V3.8 and later) if you want to turn this off
% such as if a citation ever needs to be enclosed in parenthesis.
% cite.sty is already installed on most LaTeX systems. Be sure and use
% version 5.0 (2009-03-20) and later if using hyperref.sty.
% The latest version can be obtained at:
% http://www.ctan.org/pkg/cite
% The documentation is contained in the cite.sty file itself.
%
% Note that some packages require special options to format as the Computer
% Society requires. In particular, Computer Society  papers do not use
% compressed citation ranges as is done in typical IEEE papers
% (e.g., [1]-[4]). Instead, they list every citation separately in order
% (e.g., [1], [2], [3], [4]). To get the latter we need to load the cite
% package with the nocompress option which is supported by cite.sty v4.0
% and later. Note also the use of a CLASSOPTION conditional provided by
% IEEEtran.cls V1.7 and later.





% *** GRAPHICS RELATED PACKAGES ***
%
\ifCLASSINFOpdf
   \usepackage[pdftex]{graphicx}
  % declare the path(s) where your graphic files are
   \graphicspath{{static/}}
  % and their extensions so you won't have to specify these with
  % every instance of \includegraphics
   \DeclareGraphicsExtensions{.pdf,.jpeg,.png}
\else
  % or other class option (dvipsone, dvipdf, if not using dvips). graphicx
  % will default to the driver specified in the system graphics.cfg if no
  % driver is specified.
   \usepackage[dvips]{graphicx}
  % declare the path(s) where your graphic files are
   \graphicspath{{static/}}
  % and their extensions so you won't have to specify these with
  % every instance of \includegraphics
   \DeclareGraphicsExtensions{.eps}
\fi
% graphicx was written by David Carlisle and Sebastian Rahtz. It is
% required if you want graphics, photos, etc. graphicx.sty is already
% installed on most LaTeX systems. The latest version and documentation
% can be obtained at: 
% http://www.ctan.org/pkg/graphicx
% Another good source of documentation is "Using Imported Graphics in
% LaTeX2e" by Keith Reckdahl which can be found at:
% http://www.ctan.org/pkg/epslatex
%
% latex, and pdflatex in dvi mode, support graphics in encapsulated
% postscript (.eps) format. pdflatex in pdf mode supports graphics
% in .pdf, .jpeg, .png and .mps (metapost) formats. Users should ensure
% that all non-photo figures use a vector format (.eps, .pdf, .mps) and
% not a bitmapped formats (.jpeg, .png). The IEEE frowns on bitmapped formats
% which can result in "jaggedy"/blurry rendering of lines and letters as
% well as large increases in file sizes.
%
% You can find documentation about the pdfTeX application at:
% http://www.tug.org/applications/pdftex






% *** MATH PACKAGES ***
%
%\usepackage{amsmath}
% A popular package from the American Mathematical Society that provides
% many useful and powerful commands for dealing with mathematics.
%
% Note that the amsmath package sets \interdisplaylinepenalty to 10000
% thus preventing page breaks from occurring within multiline equations. Use:
%\interdisplaylinepenalty=2500
% after loading amsmath to restore such page breaks as IEEEtran.cls normally
% does. amsmath.sty is already installed on most LaTeX systems. The latest
% version and documentation can be obtained at:
% http://www.ctan.org/pkg/amsmath





% *** SPECIALIZED LIST PACKAGES ***
%
%\usepackage{algorithmic}
% algorithmic.sty was written by Peter Williams and Rogerio Brito.
% This package provides an algorithmic environment fo describing algorithms.
% You can use the algorithmic environment in-text or within a figure
% environment to provide for a floating algorithm. Do NOT use the algorithm
% floating environment provided by algorithm.sty (by the same authors) or
% algorithm2e.sty (by Christophe Fiorio) as the IEEE does not use dedicated
% algorithm float types and packages that provide these will not provide
% correct IEEE style captions. The latest version and documentation of
% algorithmic.sty can be obtained at:
% http://www.ctan.org/pkg/algorithms
% Also of interest may be the (relatively newer and more customizable)
% algorithmicx.sty package by Szasz Janos:
% http://www.ctan.org/pkg/algorithmicx




% *** ALIGNMENT PACKAGES ***
%
%\usepackage{array}
% Frank Mittelbach's and David Carlisle's array.sty patches and improves
% the standard LaTeX2e array and tabular environments to provide better
% appearance and additional user controls. As the default LaTeX2e table
% generation code is lacking to the point of almost being broken with
% respect to the quality of the end results, all users are strongly
% advised to use an enhanced (at the very least that provided by array.sty)
% set of table tools. array.sty is already installed on most systems. The
% latest version and documentation can be obtained at:
% http://www.ctan.org/pkg/array


% IEEEtran contains the IEEEeqnarray family of commands that can be used to
% generate multiline equations as well as matrices, tables, etc., of high
% quality.




% *** SUBFIGURE PACKAGES ***
\ifCLASSOPTIONcompsoc
  \usepackage[caption=false,font=footnotesize,labelfont=sf,textfont=sf]{subfig}
\else
  \usepackage[caption=false,font=footnotesize]{subfig}
\fi
% subfig.sty, written by Steven Douglas Cochran, is the modern replacement
% for subfigure.sty, the latter of which is no longer maintained and is
% incompatible with some LaTeX packages including fixltx2e. However,
% subfig.sty requires and automatically loads Axel Sommerfeldt's caption.sty
% which will override IEEEtran.cls' handling of captions and this will result
% in non-IEEE style figure/table captions. To prevent this problem, be sure
% and invoke subfig.sty's "caption=false" package option (available since
% subfig.sty version 1.3, 2005/06/28) as this is will preserve IEEEtran.cls
% handling of captions.
% Note that the Computer Society format requires a sans serif font rather
% than the serif font used in traditional IEEE formatting and thus the need
% to invoke different subfig.sty package options depending on whether
% compsoc mode has been enabled.
%
% The latest version and documentation of subfig.sty can be obtained at:
% http://www.ctan.org/pkg/subfig




% *** FLOAT PACKAGES ***
%
%\usepackage{fixltx2e}
% fixltx2e, the successor to the earlier fix2col.sty, was written by
% Frank Mittelbach and David Carlisle. This package corrects a few problems
% in the LaTeX2e kernel, the most notable of which is that in current
% LaTeX2e releases, the ordering of single and double column floats is not
% guaranteed to be preserved. Thus, an unpatched LaTeX2e can allow a
% single column figure to be placed prior to an earlier double column
% figure.
% Be aware that LaTeX2e kernels dated 2015 and later have fixltx2e.sty's
% corrections already built into the system in which case a warning will
% be issued if an attempt is made to load fixltx2e.sty as it is no longer
% needed.
% The latest version and documentation can be found at:
% http://www.ctan.org/pkg/fixltx2e


%\usepackage{stfloats}
% stfloats.sty was written by Sigitas Tolusis. This package gives LaTeX2e
% the ability to do double column floats at the bottom of the page as well
% as the top. (e.g., "\begin{figure*}[!b]" is not normally possible in
% LaTeX2e). It also provides a command:
%\fnbelowfloat
% to enable the placement of footnotes below bottom floats (the standard
% LaTeX2e kernel puts them above bottom floats). This is an invasive package
% which rewrites many portions of the LaTeX2e float routines. It may not work
% with other packages that modify the LaTeX2e float routines. The latest
% version and documentation can be obtained at:
% http://www.ctan.org/pkg/stfloats
% Do not use the stfloats baselinefloat ability as the IEEE does not allow
% \baselineskip to stretch. Authors submitting work to the IEEE should note
% that the IEEE rarely uses double column equations and that authors should try
% to avoid such use. Do not be tempted to use the cuted.sty or midfloat.sty
% packages (also by Sigitas Tolusis) as the IEEE does not format its papers in
% such ways.
% Do not attempt to use stfloats with fixltx2e as they are incompatible.
% Instead, use Morten Hogholm'a dblfloatfix which combines the features
% of both fixltx2e and stfloats:
%
% \usepackage{dblfloatfix}
% The latest version can be found at:
% http://www.ctan.org/pkg/dblfloatfix




%\ifCLASSOPTIONcaptionsoff
%  \usepackage[nomarkers]{endfloat}
% \let\MYoriglatexcaption\caption
% \renewcommand{\caption}[2][\relax]{\MYoriglatexcaption[#2]{#2}}
%\fi
% endfloat.sty was written by James Darrell McCauley, Jeff Goldberg and 
% Axel Sommerfeldt. This package may be useful when used in conjunction with 
% IEEEtran.cls'  captionsoff option. Some IEEE journals/societies require that
% submissions have lists of figures/tables at the end of the paper and that
% figures/tables without any captions are placed on a page by themselves at
% the end of the document. If needed, the draftcls IEEEtran class option or
% \CLASSINPUTbaselinestretch interface can be used to increase the line
% spacing as well. Be sure and use the nomarkers option of endfloat to
% prevent endfloat from "marking" where the figures would have been placed
% in the text. The two hack lines of code above are a slight modification of
% that suggested by in the endfloat docs (section 8.4.1) to ensure that
% the full captions always appear in the list of figures/tables - even if
% the user used the short optional argument of \caption[]{}.
% IEEE papers do not typically make use of \caption[]'s optional argument,
% so this should not be an issue. A similar trick can be used to disable
% captions of packages such as subfig.sty that lack options to turn off
% the subcaptions:
% For subfig.sty:
% \let\MYorigsubfloat\subfloat
% \renewcommand{\subfloat}[2][\relax]{\MYorigsubfloat[]{#2}}
% However, the above trick will not work if both optional arguments of
% the \subfloat command are used. Furthermore, there needs to be a
% description of each subfigure *somewhere* and endfloat does not add
% subfigure captions to its list of figures. Thus, the best approach is to
% avoid the use of subfigure captions (many IEEE journals avoid them anyway)
% and instead reference/explain all the subfigures within the main caption.
% The latest version of endfloat.sty and its documentation can obtained at:
% http://www.ctan.org/pkg/endfloat
%
% The IEEEtran \ifCLASSOPTIONcaptionsoff conditional can also be used
% later in the document, say, to conditionally put the References on a 
% page by themselves.




% *** PDF, URL AND HYPERLINK PACKAGES ***
%
%\usepackage{url}
% url.sty was written by Donald Arseneau. It provides better support for
% handling and breaking URLs. url.sty is already installed on most LaTeX
% systems. The latest version and documentation can be obtained at:
% http://www.ctan.org/pkg/url
% Basically, \url{my_url_here}.





% *** Do not adjust lengths that control margins, column widths, etc. ***
% *** Do not use packages that alter fonts (such as pslatex).         ***
% There should be no need to do such things with IEEEtran.cls V1.6 and later.
% (Unless specifically asked to do so by the journal or conference you plan
% to submit to, of course. )


% correct bad hyphenation here
\hyphenation{op-tical net-works semi-conduc-tor}


\begin{document}
%
% paper title
% Titles are generally capitalized except for words such as a, an, and, as,
% at, but, by, for, in, nor, of, on, or, the, to and up, which are usually
% not capitalized unless they are the first or last word of the title.
% Linebreaks \\ can be used within to get better formatting as desired.
% Do not put math or special symbols in the title.
\title{Técnicas de Aprendizagem Profunda na Análise Comportamental de
  Motoristas: \\
  Revisão Sistemática da Literatura}
%
%
% author names and IEEE memberships
% note positions of commas and nonbreaking spaces ( ~ ) LaTeX will not break
% a structure at a ~ so this keeps an author's name from being broken across
% two lines.
% use \thanks{} to gain access to the first footnote area
% a separate \thanks must be used for each paragraph as LaTeX2e's \thanks
% was not built to handle multiple paragraphs
%
%
%\IEEEcompsocitemizethanks is a special \thanks that produces the bulleted
% lists the Computer Society journals use for "first footnote" author
% affiliations. Use \IEEEcompsocthanksitem which works much like \item
% for each affiliation group. When not in compsoc mode,
% \IEEEcompsocitemizethanks becomes like \thanks and
% \IEEEcompsocthanksitem becomes a line break with idention. This
% facilitates dual compilation, although admittedly the differences in the
% desired content of \author between the different types of papers makes a
% one-size-fits-all approach a daunting prospect. For instance, compsoc 
% journal papers have the author affiliations above the "Manuscript
% received ..."  text while in non-compsoc journals this is reversed. Sigh.

\author{Lucas~Lima}
% note the % following the last \IEEEmembership and also \thanks - 
% these prevent an unwanted space from occurring between the last author name
% and the end of the author line. i.e., if you had this:
% 
% \author{....lastname \thanks{...} \thanks{...} }
%                     ^------------^------------^----Do not want these spaces!
%
% a space would be appended to the last name and could cause every name on that
% line to be shifted left slightly. This is one of those "LaTeX things". For
% instance, "\textbf{A} \textbf{B}" will typeset as "A B" not "AB". To get
% "AB" then you have to do: "\textbf{A}\textbf{B}"
% \thanks is no different in this regard, so shield the last } of each \thanks
% that ends a line with a % and do not let a space in before the next \thanks.
% Spaces after \IEEEmembership other than the last one are OK (and needed) as
% you are supposed to have spaces between the names. For what it is worth,
% this is a minor point as most people would not even notice if the said evil
% space somehow managed to creep in.



% The paper headers
%\markboth{Journal of \LaTeX\ Class Files,~Vol.~14, No.~8, August~2015}%
%{Shell \MakeLowercase{\textit{et al.}}: Bare Demo of IEEEtran.cls for Computer Society Journals}
% The only time the second header will appear is for the odd numbered pages
% after the title page when using the twoside option.
% 
% *** Note that you probably will NOT want to include the author's ***
% *** name in the headers of peer review papers.                   ***
% You can use \ifCLASSOPTIONpeerreview for conditional compilation here if
% you desire.



% The publisher's ID mark at the bottom of the page is less important with
% Computer Society journal papers as those publications place the marks
% outside of the main text columns and, therefore, unlike regular IEEE
% journals, the available text space is not reduced by their presence.
% If you want to put a publisher's ID mark on the page you can do it like
% this:
%\IEEEpubid{0000--0000/00\$00.00~\copyright~2015 IEEE}
% or like this to get the Computer Society new two part style.
%\IEEEpubid{\makebox[\columnwidth]{\hfill 0000--0000/00/\$00.00~\copyright~2015 IEEE}%
%\hspace{\columnsep}\makebox[\columnwidth]{Published by the IEEE Computer Society\hfill}}
% Remember, if you use this you must call \IEEEpubidadjcol in the second
% column for its text to clear the IEEEpubid mark (Computer Society jorunal
% papers don't need this extra clearance.)



% use for special paper notices
%\IEEEspecialpapernotice{(Invited Paper)}



% for Computer Society papers, we must declare the abstract and index terms
% PRIOR to the title within the \IEEEtitleabstractindextext IEEEtran
% command as these need to go into the title area created by \maketitle.
% As a general rule, do not put math, special symbols or citations
% in the abstract or keywords.
\renewcommand{\abstractname}{Resumo}
\IEEEtitleabstractindextext{%
  \begin{abstract}
    As técnicas de \textit{deep learning} e a sua capacidade de
    aprender tarefas complexas a partir de conjuntos de dados
    complexos e com grande dimensão, têm atraído cada vez mais atenção
    e adeptos. A análise comportamental, por ser uma tarefa que exige
    a identificação de correlações entre os aspectos analisados,
    torna-se uma candidata natural para a aplicação desta
    técnica. Neste artigo é feita uma revisão sistemática da
    literatura com o objetivo de identificar estudos que realizam a
    tarefa de análise comportamental em conjunto com outras diretrizes
    que regem a aplicação desta técnica.
  \end{abstract}

% Note that keywords are not normally used for peerreview papers.
%\renewcommand{\IEEEkeywordsname}{Palavras-chave}
\begin{IEEEkeywords}
Motorista, Análise Comportamental, Deep Learning, Aprendizagem Profunda
\end{IEEEkeywords}}


% make the title area
\maketitle


% To allow for easy dual compilation without having to reenter the
% abstract/keywords data, the \IEEEtitleabstractindextext text will
% not be used in maketitle, but will appear (i.e., to be "transported")
% here as \IEEEdisplaynontitleabstractindextext when the compsoc 
% or transmag modes are not selected <OR> if conference mode is selected 
% - because all conference papers position the abstract like regular
% papers do.
\IEEEdisplaynontitleabstractindextext
% \IEEEdisplaynontitleabstractindextext has no effect when using
% compsoc or transmag under a non-conference mode.



% For peer review papers, you can put extra information on the cover
% page as needed:
% \ifCLASSOPTIONpeerreview
% \begin{center} \bfseries EDICS Category: 3-BBND \end{center}
% \fi
%
% For peerreview papers, this IEEEtran command inserts a page break and
% creates the second title. It will be ignored for other modes.
\IEEEpeerreviewmaketitle



\IEEEraisesectionheading{\section{Introdução}\label{sec:introduction}}
\nocite{*}
% Computer Society journal (but not conference!) papers do something unusual
% with the very first section heading (almost always called "Introduction").
% They place it ABOVE the main text! IEEEtran.cls does not automatically do
% this for you, but you can achieve this effect with the provided
% \IEEEraisesectionheading{} command. Note the need to keep any \label that
% is to refer to the section immediately after \section in the above as
% \IEEEraisesectionheading puts \section within a raised box.




% The very first letter is a 2 line initial drop letter followed
% by the rest of the first word in caps (small caps for compsoc).
% 
% form to use if the first word consists of a single letter:
% \IEEEPARstart{A}{demo} file is ....
% 
% form to use if you need the single drop letter followed by
% normal text (unknown if ever used by the IEEE):
% \IEEEPARstart{A}{}demo file is ....
% 
% Some journals put the first two words in caps:
% \IEEEPARstart{T}{his demo} file is ....
% 
% Here we have the typical use of a "T" for an initial drop letter
% and "HIS" in caps to complete the first word.
% You must have at least 2 lines in the paragraph with the drop letter
% (should never be an issue)
% needed in second column of first page if using \IEEEpubid
%\IEEEpubidadjcol

% \IEEEPARstart{A}{revisão} sistemática de literatura, apesar de ser um trabalho extenso
% e árduo, é útil para mapear e entender o estado da arte sobre uma área
% ou tema específico. Neste trabalho, a revisão sistemática de literatura auxiliou no
% mapeamento de trabalhos onde houve a aplicação de técnicas exclusivas
% sobre uma área do conhecimento específica: a utilização de métodos de
% aprendizagem profunda, aplicados na análise
% comportamental de motoristas.\par

\IEEEPARstart{A}{análise} comportamental, dentro do campo da inteligência artificial, é uma
atividade bastante difundida e pesquisada, explorando o uso de diversas técnicas de aprendizado,
como o aprendizado superficial, \textit{shallow learning}, a lógica
difusa e o aprendizado profundo, mais conhecido como \textit{deep
  learning}. \textit{Deep learning} é um método que aplica
técnicas de aprendizado superficial não lineares encadeadas em múltiplos níveis, com a finalidade de aprender tarefas complexas a partir de dados não
tratados, LeCun \emph{et~al.}\cite{Lecun2015436}.

O comportamento de condução pode ser definido como uma sequência de
ações segmentáveis e realizadas uma a uma quando há interação entre o motorista e o
veículo \cite{Liu2016}. Nesta revisão
utilizamos a definição acima, mas com uma especifidade ainda maior,
além de ser uma sequência de ações, executadas uma a uma, elas devem
ocorrer por um período de tempo limitado e ser mensuráveis por meio de
sensores automotivos instalados no veículo.

As sessões seguintes estruturam esta pesquisa conforme a ordem: na seção
\ref{sec:protocolo} o protocolo da pesquisa é detalhado, bem como seu
objetivo e as questões de pesquisa; na seção \ref{sec:resultados}
são apresentados os resultados obtidos; a discussão sobre o tema
ocorre na seção \ref{sec:discussao}; e a seção \ref{sec:conclusao} é
responsável pela conclusão deste trabalho.

\section{Protocolo de revisão}
\label{sec:protocolo}
Nesta seção são apresentadas todas as diretrizes que guiaram o
processo de revisão sistemática da literatura, a fim de atingir os
objetos explicitados na seção \ref{sec:protocolo:objetivos}.

O modelo de trabalho foi baseado nas diretrizes
documentadas pela Kitchenham \cite{kitchenham2007} e executado
seguindo três macro etapas:
\begin{enumerate*}
\item Levantamento bibliográfico;
\item Seleção dos artigos;
\item Análise dos estudos selecionados
\end{enumerate*}. A imagem \ref{fig:etapas} demonstra a sequência iterativa de
atividades realizadas dentro de cada macro etapa.

O levantamento bibliográfico foi essencial na definição e refinamento da
\textit{string} de busca, discutida com maiores detalhes no item
\ref{sec:protocolo:string}; e o uso dos filtros de pesquisa serviram
para delimitar a abrangência e destacar os resultados que seriam
melhor aproveitados para o cumprimento do objetivo deste trabalho. A etapa
de seleção encarregou-se da curadoria do material gerado pelo estágio
anterior. Àqueles artigos que não infringiram nenhum dos critérios de
exclusão foram reservados para uma análise mais minuciosa e crítica. A
conclusão da última macro etapa que consistiu na análise crítica e mapeamento das
técnicas de aprendizagem profunda na análise comportamental de
motoristas, resultou na seção \ref{sec:discussao}.

\begin{figure*}[!t]
\centering
\includegraphics[width=\textwidth, height=6cm]{etapas14}
\caption{Detalhamento das três macro etapas utilizadas para a
  confecção desta revisão sistemática da literatura.}
\label{fig:etapas}
\end{figure*}

\subsection{Objetivos da revisão sistemática de literatura}
\label{sec:protocolo:objetivos}
A revisão sistemática da literatura tem como objetivo coletar,
organizar e compreender o estado da arte de projetos de pesquisa que
explorem exclusivamente a aplicação de métodos de aprendizagem
profunda na análise idiossincrática de condutores veiculares. Além
destas características, o estudo limita-se a incluir pesquisas cujo os
dados coletados sejam provenientes de sensores instalados no veículo.

Além de propiciar avanços significativos nos campos de reconhecimento
de imagens e reconhecimento da fala, a aplicação
de \textit{deep learning} também traz melhores resultados quando
aplicada em conjuntos de dados grandes e de alta dimensionalidade,
se comparada aos métodos tradicionais de aprendizado superficial
\cite{Lecun2015436}. Para o objetivo do estudo, esta técnica é particularmente
importante, pois as informações que descrevem o comportamento do condutor são
classificadas como séries temporais multidimensionais
\cite{Liu2018}. Aplicada a este contexto, a série de dados descreve o estado atual do veículo, dado um momento
específico no tempo, e é composta por medições de diversos sensores
como a aceleração, a velocidade, o ângulo do volante, dentre outras mais.

\subsection{\textit{String} de busca}
\label{sec:protocolo:string}
A \textit{string} de busca constitui uma das principais partes de uma
revisão sistemática da literatura, pois é a partir dela que todo o
estudo será conduzido. Caso o conjunto de termos escolhidos não
consiga expressar plenamente a necessidade da pesquisa, os resultados
oriundos das bases de pesquisa não estarão condizentes com as necessidades do
pesquisador. Justamente para evitar o cenário acima, a cadeia de pesquisa
empregada neste estudo foi dividida e evoluída iterativamente com o
auxílio das plataformas Scopus e Web of Science.

As duas partes da
\textit{string} passaram pelo mesmo processo de amadurecimento. Primeiro um termo genérico foi apresentado às plataformas
supracitadas e, a partir das palavras-chaves retornadas, novos termos
foram sendo incorporados à cadeia original. A quantidade total de documentos
retornados e a quantidade de documentos por área do conhecimento foram
as métricas utilizadas para avaliar a inclusão de novos
termos.

A \textit{query} de busca pelas técnicas de aprendizagem profunda
contém 30 termos. Sem a aplicação de nenhum filtro nos resultados, 43.272 artigos foram
retornados pelo Scopus e 24.928 pela Web of Science, uma diferença de
57,60\%. Já a busca pelo comportamento do motorista possui 45
termos ao todo e, também sem a aplicação de filtros, a busca nas
mesmas bases retornaram 36.968 e 21.548 artigos, respectivamente. Ambas as \textit{queries} podem ser
consultadas nas tabelas \ref{algo:deeplearning} e
\ref{algo:driverbehavior}.

Quando a junção das duas cadeias é feita pelo conector lógico
\emph{\textbf{E}}, surge uma nova \textit{string} de busca com 75 termos. Quando esta nova
\textit{string} é apresentada às plataformas de pesquisa, os
resultados são reduzidos para 161 artigos no Scopus e 83 na Web of
Science.

\begin{algorithm}
  \label{algo:deeplearning}
  \renewcommand{\algorithmcfname}{String de busca}
  \caption{Busca por técnicas de \textit{deep learning} considerando o
  título, resumo e palavras-chave}
  \SetAlgoLined
  \DontPrintSemicolon
  \FuncSty{TITLE-ABS-KEY:}\;
  \Indp
  \FuncArgSty{deep learning} \ArgSty{OR} \ArgSty{convolutional
    neural network}  \ArgSty{OR}  \ArgSty{convolution* network}
  \ArgSty{OR}  \ArgSty{deep neural network}  \ArgSty{OR}  \ArgSty{auto
    encoder}  \ArgSty{OR}  \ArgSty{deep belief network}  \ArgSty{OR}
  \ArgSty{convolutional network}  \ArgSty{OR}  \ArgSty{cnn}
  \ArgSty{OR}  \ArgSty{dbn}  \ArgSty{OR}  \ArgSty{deep architecture}
  \ArgSty{OR}  \ArgSty{autoencoder}  \ArgSty{OR}  \ArgSty{deep
    bayesian network}  \ArgSty{OR}  \ArgSty{deep * network}
  \ArgSty{OR}  \ArgSty{deep convolution network}  \ArgSty{OR}
  \ArgSty{deep convolutional network}  \ArgSty{OR}  \ArgSty{deep
    neural convolutional network}  \ArgSty{OR}  \ArgSty{deep neural
    convolution network}  \ArgSty{OR}  \ArgSty{deep autoencoder}
  \ArgSty{OR}  \ArgSty{deep auto encoder}  \ArgSty{OR}  \ArgSty{deep *
    autoencoder}  \ArgSty{OR}  \ArgSty{deep * auto encoder}
  \ArgSty{OR}  \ArgSty{dnn}  \ArgSty{OR}  \ArgSty{deep multilayer
    neural network}  \ArgSty{OR}  \ArgSty{deep artificial neural
    network}  \ArgSty{OR}  \ArgSty{deep boltzmann machine}
  \ArgSty{OR}  \ArgSty{deep multitask learning}  \ArgSty{OR}
  \ArgSty{deep extreme learning machine}  \ArgSty{OR}  \ArgSty{deep
    recurrent neural network}  \ArgSty{OR}  \ArgSty{deep rnn}
  \ArgSty{OR}  \ArgSty{deep fuzzy}\;
\end{algorithm}

\begin{algorithm}
  \label{algo:driverbehavior}
  \renewcommand{\algorithmcfname}{String de busca}
  \caption{Busca por comportamento do motorista (\textit{driving
      behavior}) e variações, considerando o
  título, resumo e palavras-chave}
  \SetAlgoLined
  \DontPrintSemicolon
  \FuncSty{TITLE-ABS-KEY:}\;
  \Indp
  \FuncArgSty{drive* identif*}  \ArgSty{OR}  \FuncArgSty{drive* fingerprint*}  \ArgSty{OR}  \FuncArgSty{drive* recogn*}  \ArgSty{OR}  \FuncArgSty{drive* behaviour*}  \ArgSty{OR}  \FuncArgSty{drive* behavior*}  \ArgSty{OR}  \FuncArgSty{drive* model*}  \ArgSty{OR}  \FuncArgSty{drive* perform*}  \ArgSty{OR}  \FuncArgSty{drive* modelling}  \ArgSty{OR}  \FuncArgSty{drive* modeling}  \ArgSty{OR}  \FuncArgSty{drive* feature}  \ArgSty{OR}  \FuncArgSty{drive* mapping}  \ArgSty{OR}  \FuncArgSty{drive* characteristic}  \ArgSty{OR}  \FuncArgSty{drive* characteristical}  \ArgSty{OR}  \FuncArgSty{drive* trait*}  \ArgSty{OR}  \FuncArgSty{conductor* identif*}  \ArgSty{OR}  \FuncArgSty{conductor* fingerprint*}  \ArgSty{OR}  \FuncArgSty{conductor* recogn*}  \ArgSty{OR}  \FuncArgSty{conductor* behaviour*}  \ArgSty{OR}  \FuncArgSty{conductor* behavior*}  \ArgSty{OR}  \FuncArgSty{conductor* model*}  \ArgSty{OR}  \FuncArgSty{conductor* perform*}  \ArgSty{OR}  \FuncArgSty{conductor* modelling}  \ArgSty{OR}  \FuncArgSty{conductor* modeling}  \ArgSty{OR}  \FuncArgSty{conductor* feature}  \ArgSty{OR}  \FuncArgSty{conductor* mapping}  \ArgSty{OR}  \FuncArgSty{conductor* characteristic}  \ArgSty{OR}  \FuncArgSty{conductor* characteristical}  \ArgSty{OR}  \FuncArgSty{conductor* trait*}  \ArgSty{OR}  \FuncArgSty{driving identif*}  \ArgSty{OR}  \FuncArgSty{driving fingerprint*}  \ArgSty{OR}  \FuncArgSty{driving recogn*}  \ArgSty{OR}  \FuncArgSty{driving behaviour*}  \ArgSty{OR}  \FuncArgSty{driving behavior*}  \ArgSty{OR}  \FuncArgSty{driving model*}  \ArgSty{OR}  \FuncArgSty{driving perform*}  \ArgSty{OR}  \FuncArgSty{driving modelling}  \ArgSty{OR}  \FuncArgSty{driving modeling}  \ArgSty{OR}  \FuncArgSty{driving feature}  \ArgSty{OR}  \FuncArgSty{driving mapping}  \ArgSty{OR}  \FuncArgSty{driving characteristic}  \ArgSty{OR}  \FuncArgSty{driving characteristical}  \ArgSty{OR}  \FuncArgSty{driving trait*}  \ArgSty{OR}  \FuncArgSty{driving abilit*}  \ArgSty{OR}  \FuncArgSty{driver* abilit*}  \ArgSty{OR}  \FuncArgSty{conductor* abilit*}\;
\end{algorithm}


\subsection{Questões de pesquisa}
\label{sec:protocolo:questoes}
Kitchenham \cite{kitchenham2007} defende que a definição das questões
de pesquisa é o elemento mais importante de qualquer revisão
sistemática e isto ocorre porque toda a metodologia da revisão será guiada
por estes questionamentos.

No presente estudo foram elaboradas três questões de pesquisa com o
objetivo de identificar quais técnicas de aprendizado profundo,
aplicadas ao contexto de análise comportamental de condutores, vem
sendo utilizadas e quais são as suas vantagens e desvantagens frente à
outros modelos. Também houve a preocupação em compreender como os
atributos captados do veículo instantaneamente estavam sendo tratados,
organizados e apresentados à rede neural. As questões de pesquisa
(\emph{QP}) que nortearam este estudo, e suas justificativas,
encontram-se abaixo ordenadas por relevância:

\begin{description}
\item [\emph{QP1: }]\emph{Quais são as arquiteturas de \textit{deep learning} aplicadas
  na área de reconhecimento do comportamento de motoristas?}
  \vspace{0.15cm}
  \begin{adjustwidth}{}{}
    As técnicas de aprendizagem profunda são aplicadas em diversas
    áreas do conhecimento. Wang \cite{Wang20171360},  utiliza
    \textit{deep learning} para identificar falhas em turbinas
    eólicas; Mukherjee \cite{Mukherjee20171205} aplica a mesma técnica para realizar
    segmentação de nódulos pulmonares. As técnicas de redes neurais profundas,
    assim como as técnicas de redes neurais superficiais, podem ser classificadas
    como \textit{feedforward} ou recorrentes; supervisionadas, semi-supervisionadas, não
    supervisionadas ou de reforço.

    Entender a arquitetura utilizada na rede neural, e a sua
    classificação, é importante para identificar quais redes produzem
    os melhores resultados considerando os dados apresentados e o
    tipo de problema que ela foi proposta à resolver.
  \end{adjustwidth}
  \vspace{0.3cm}
\item [\emph{QP2: }]\emph{Como devem ser tratados os casos de \textit{outliers} e
  medições que apresentam interferências ou dados corrompidos?}
  \vspace{0.15cm}
  \begin{adjustwidth}{}{}
    Na atualidade, a captação de dados sensoriados pode ser
    considerada uma tarefa trivial, por isso os sensores estão presentes nos mais
    variados objetos do cotidiano. Mesmo se tratando de uma atividade
    simples, existe a possibilidade de falha na leitura, produção de dados
    ruidosos ou corrompimento das informações. Liu \cite{Liu2016}
    alerta que estes dados defeituosos interferem na
    capacidade de segmentação do comportamento do motorista.
    
    Esta pergunta auxiliou a identificar e entender quais são as estratégias mais
    utilizadas e adequadas na minimização do impacto causado por dados defeituosos e \textit{outliers}.
  \end{adjustwidth}
    \vspace{0.3cm}
\item [\emph{QP3: }]\emph{Quais campos sensoriados devem ser
    apresentados às redes neurais de aprendizado profundo?}
    \vspace{0.15cm}
    \begin{adjustwidth}{}{}
      O objetivo desta pergunta é entender se a ordem de apresentação
      dos dados coletados pelos diferentes sensores acoplados ao
      veículo, altera o desempenho da rede \textit{deep learning} durante
      a execução da atividade de segmentação de características do motorista.
    \end{adjustwidth}
\end{description}

\subsection{Definição das fontes de pesquisa e filtros de resultado}
A escolha das fontes de pesquisa ocorreu durante a primeira macro
etapa deste projeto, a etapa de levantamento bibliográfico. A escolha
das fontes de pesquisa não pode ser banalizada ou tratada com menor
importância, pois elas serão responsáveis por entregar o corpo de
textos a serem trabalhados nas demais etapas da revisão sistemática de
literatura.

Na composição deste trabalho foram escolhidas duas bases de dados, a Scopus e a Web of
Science. Outras fontes foram descartadas devido ao número de estudos
retornados ser inferior, quando comparado às fontes de pesquisa previamente citadas, e por apresentarem artigos que já haviam
sido mapeados.

Após a execução da \textit{string} de busca nas duas fontes de
pesquisa selecionadas, foi necessário refinar os resultados aplicando-lhes
filtros restritivos. Primeiramente a área do conhecimento foi
restringida à Ciência da Computação, Engenharia, Matemática e Ciências
da Decisão; o
segundo passo foi desconsiderar estudos que não estivessem na língua
inglesa; finalmente, foram desprezados artigos que não eram
classificados como artigos de conferência, artigos e artigos na
imprensa. Não foi preciso restringir a procura à um período específico
de tempo porque, para este estudo, era importante entender a evolução
do uso das técnicas de aprendizado profundo.

Os parâmetros de busca tanto do Scopus quanto da Web of
Science estão disponíveis integralmente em \ref{algo:buscascopus} e
\ref{algo:buscawos}, respectivamente.

\begin{algorithm*}
  \label{algo:buscascopus}
  \renewcommand{\algorithmcfname}{String de busca}
  \caption{\textit{String} de busca específica utilizada na plataforma Scopus
    para pesquisar por \textit{deep learning} e comportamento do motorista (\textit{driving
      behavior}), considerando variações linguísticas. A pesquisa é
    realizada considerando o título, o resumo e as palavras-chave}
  \SetAlgoLined
  \DontPrintSemicolon
  (\;
  \Indp
  \FuncSty{TITLE-ABS-KEY} (\FuncArgSty{deep learning} \ArgSty{OR} \FuncArgSty{convolutional
  neural network} \ArgSty{OR} \FuncArgSty{convolution * network}
\ArgSty{OR} \FuncArgSty{deep neural network} \ArgSty{OR}
\FuncArgSty{auto encoder} \ArgSty{OR} \FuncArgSty{deep belief network}
\ArgSty{OR} \FuncArgSty{convolutional network} \ArgSty{OR}
\FuncArgSty{cnn} \ArgSty{OR} \FuncArgSty{dbn} \ArgSty{OR}
\FuncArgSty{deep architecture} \ArgSty{OR} \FuncArgSty{autoencoder}
\ArgSty{OR} \FuncArgSty{deep bayesian network} \ArgSty{OR}
\FuncArgSty{deep * network} \ArgSty{OR} \FuncArgSty{deep convolution
  network} \ArgSty{OR} \FuncArgSty{deep convolutional network}
\ArgSty{OR} \FuncArgSty{deep neural convolutional network} \ArgSty{OR}
\FuncArgSty{deep neural convolution network} \ArgSty{OR}
\FuncArgSty{deep autoencoder} \ArgSty{OR} \FuncArgSty{deep auto
  encoder} \ArgSty{OR} \FuncArgSty{deep * autoencoder} \ArgSty{OR}
\FuncArgSty{deep * auto encoder} \ArgSty{OR} \FuncArgSty{dnn}
\ArgSty{OR} \FuncArgSty{deep multilayer neural network} \ArgSty{OR}
\FuncArgSty{deep artificial neural network} \ArgSty{OR}
\FuncArgSty{deep boltzmann machine} \ArgSty{OR} \FuncArgSty{deep
  multitask learning} \ArgSty{OR} \FuncArgSty{deep extreme learning
  machine} \ArgSty{OR} \FuncArgSty{deep recurrent neural network}
\ArgSty{OR} \FuncArgSty{deep rnn} \ArgSty{OR} \FuncArgSty{deep fuzzy}
) \;
\ArgSty{AND}\;
\FuncSty{TITLE-ABS-KEY} (\FuncArgSty{drive* identif*} \ArgSty{OR} \FuncArgSty{drive*
  fingerprint*} \ArgSty{OR} \FuncArgSty{drive* recogn*} \ArgSty{OR}
\FuncArgSty{drive* behaviour*} \ArgSty{OR} \FuncArgSty{drive*
  behavior*} \ArgSty{OR} \FuncArgSty{drive* model*} \ArgSty{OR}
\FuncArgSty{drive* perform*} \ArgSty{OR} \FuncArgSty{drive* modelling}
\ArgSty{OR} \FuncArgSty{drive* modeling} \ArgSty{OR}
\FuncArgSty{drive* feature} \ArgSty{OR} \FuncArgSty{drive* mapping}
\ArgSty{OR}  \FuncArgSty{drive* characteristic} \ArgSty{OR}
\FuncArgSty{drive* characteristical} \ArgSty{OR} \FuncArgSty{drive*
  trait*} \ArgSty{OR} \FuncArgSty{conductor* identif*} \ArgSty{OR}
\FuncArgSty{conductor* fingerprint*} \ArgSty{OR}
\FuncArgSty{conductor* recogn*} \ArgSty{OR} \FuncArgSty{conductor*
  behaviour*} \ArgSty{OR} \FuncArgSty{conductor* behavior*}
\ArgSty{OR} \FuncArgSty{conductor* model*} \ArgSty{OR}
\FuncArgSty{conductor* perform*} \ArgSty{OR} \FuncArgSty{conductor*
  modelling} \ArgSty{OR} \FuncArgSty{conductor* modeling} \ArgSty{OR}
\FuncArgSty{conductor* feature} \ArgSty{OR} \FuncArgSty{conductor*
  mapping} \ArgSty{OR} \FuncArgSty{conductor* characteristic}
\ArgSty{OR} \FuncArgSty{conductor* characteristical} \ArgSty{OR}
\FuncArgSty{conductor* trait*} \ArgSty{OR} \FuncArgSty{driving
  identif*} \ArgSty{OR} \FuncArgSty{driving fingerprint*} \ArgSty{OR}
\FuncArgSty{driving recogn*} \ArgSty{OR} \FuncArgSty{driving
  behaviour*} \ArgSty{OR} \FuncArgSty{driving behavior*} \ArgSty{OR}
\FuncArgSty{driving model*} \ArgSty{OR} \FuncArgSty{driving perform*}
\ArgSty{OR} \FuncArgSty{driving modelling} \ArgSty{OR}
\FuncArgSty{driving modeling} \ArgSty{OR} \FuncArgSty{driving feature}
\ArgSty{OR} \FuncArgSty{driving mapping} \ArgSty{OR}
\FuncArgSty{driving characteristic} \ArgSty{OR} \FuncArgSty{driving
  characteristical} \ArgSty{OR} \FuncArgSty{driving trait*}
\ArgSty{OR} \FuncArgSty{driving abilit*} \ArgSty{OR}
\FuncArgSty{driver* abilit*} \ArgSty{OR} \FuncArgSty{conductor*
  abilit*} )\;
\Indm
)\;
\ArgSty{AND}  (\;
\Indp
\FuncArgSty{LIMIT-TO ( SUBJAREA ,  "COMP" )} \ArgSty{OR}
\FuncArgSty{LIMIT-TO ( SUBJAREA ,  "ENGI" )} \ArgSty{OR}
\FuncArgSty{LIMIT-TO ( SUBJAREA ,  "MATH" )} \ArgSty{OR}
\FuncArgSty{LIMIT-TO ( SUBJAREA ,  "DECI" )} ) \ArgSty{AND} (
\FuncArgSty{EXCLUDE ( DOCTYPE ,  "re" )} ) \ArgSty{AND} (
\FuncArgSty{EXCLUDE ( DOCTYPE ,  "ch" )} ) \ArgSty{AND} (
\FuncArgSty{EXCLUDE ( DOCTYPE ,  "cr" )} )\;
\Indm)
\end{algorithm*}

\begin{algorithm*}
  \label{algo:buscawos}
  \renewcommand{\algorithmcfname}{String de busca}
  \caption{\textit{String} de busca específica utilizada na plataforma Web of Science
    para pesquisar por \textit{deep learning} e comportamento do motorista (\textit{driving
      behavior}), considerando variações linguísticas. A pesquisa é
    realizada considerando o tópico e título}
  \SetAlgoLined
  \DontPrintSemicolon
  (\;
  \Indp
  \FuncSty{TS=}(\FuncArgSty{deep learning} \ArgSty{OR} \FuncArgSty{convolutional
  neural network} \ArgSty{OR} \FuncArgSty{convolution * network}
\ArgSty{OR} \FuncArgSty{deep neural network} \ArgSty{OR}
\FuncArgSty{auto encoder} \ArgSty{OR} \FuncArgSty{deep belief network}
\ArgSty{OR} \FuncArgSty{convolutional network} \ArgSty{OR}
\FuncArgSty{cnn} \ArgSty{OR} \FuncArgSty{dbn} \ArgSty{OR}
\FuncArgSty{deep architecture} \ArgSty{OR} \FuncArgSty{autoencoder}
\ArgSty{OR} \FuncArgSty{deep bayesian network} \ArgSty{OR}
\FuncArgSty{deep * network} \ArgSty{OR} \FuncArgSty{deep convolution
  network} \ArgSty{OR} \FuncArgSty{deep convolutional network}
\ArgSty{OR} \FuncArgSty{deep neural convolutional network} \ArgSty{OR}
\FuncArgSty{deep neural convolution network} \ArgSty{OR}
\FuncArgSty{deep autoencoder} \ArgSty{OR} \FuncArgSty{deep auto
  encoder} \ArgSty{OR} \FuncArgSty{deep * autoencoder} \ArgSty{OR}
\FuncArgSty{deep * auto encoder} \ArgSty{OR} \FuncArgSty{dnn}
\ArgSty{OR} \FuncArgSty{deep multilayer neural network} \ArgSty{OR}
\FuncArgSty{deep artificial neural network} \ArgSty{OR}
\FuncArgSty{deep boltzmann machine} \ArgSty{OR} \FuncArgSty{deep
  multitask learning} \ArgSty{OR} \FuncArgSty{deep extreme learning
  machine} \ArgSty{OR} \FuncArgSty{deep recurrent neural network}
\ArgSty{OR} \FuncArgSty{deep rnn} \ArgSty{OR} \FuncArgSty{deep fuzzy}
)\;
\ArgSty{AND}\;
\FuncSty{TS=}(\FuncArgSty{drive* identif*} \ArgSty{OR} \FuncArgSty{drive*
  fingerprint*} \ArgSty{OR} \FuncArgSty{drive* recogn*} \ArgSty{OR}
\FuncArgSty{drive* behaviour*} \ArgSty{OR} \FuncArgSty{drive*
  behavior*} \ArgSty{OR} \FuncArgSty{drive* model*} \ArgSty{OR}
\FuncArgSty{drive* perform*} \ArgSty{OR} \FuncArgSty{drive* modelling}
\ArgSty{OR} \FuncArgSty{drive* modeling} \ArgSty{OR}
\FuncArgSty{drive* feature} \ArgSty{OR} \FuncArgSty{drive* mapping}
\ArgSty{OR}  \FuncArgSty{drive* characteristic} \ArgSty{OR}
\FuncArgSty{drive* characteristical} \ArgSty{OR} \FuncArgSty{drive*
  trait*} \ArgSty{OR} \FuncArgSty{conductor* identif*} \ArgSty{OR}
\FuncArgSty{conductor* fingerprint*} \ArgSty{OR}
\FuncArgSty{conductor* recogn*} \ArgSty{OR} \FuncArgSty{conductor*
  behaviour*} \ArgSty{OR} \FuncArgSty{conductor* behavior*}
\ArgSty{OR} \FuncArgSty{conductor* model*} \ArgSty{OR}
\FuncArgSty{conductor* perform*} \ArgSty{OR} \FuncArgSty{conductor*
  modelling} \ArgSty{OR} \FuncArgSty{conductor* modeling} \ArgSty{OR}
\FuncArgSty{conductor* feature} \ArgSty{OR} \FuncArgSty{conductor*
  mapping} \ArgSty{OR} \FuncArgSty{conductor* characteristic}
\ArgSty{OR} \FuncArgSty{conductor* characteristical} \ArgSty{OR}
\FuncArgSty{conductor* trait*} \ArgSty{OR} \FuncArgSty{driving
  identif*} \ArgSty{OR} \FuncArgSty{driving fingerprint*} \ArgSty{OR}
\FuncArgSty{driving recogn*} \ArgSty{OR} \FuncArgSty{driving
  behaviour*} \ArgSty{OR} \FuncArgSty{driving behavior*} \ArgSty{OR}
\FuncArgSty{driving model*} \ArgSty{OR} \FuncArgSty{driving perform*}
\ArgSty{OR} \FuncArgSty{driving modelling} \ArgSty{OR}
\FuncArgSty{driving modeling} \ArgSty{OR} \FuncArgSty{driving feature}
\ArgSty{OR} \FuncArgSty{driving mapping} \ArgSty{OR}
\FuncArgSty{driving characteristic} \ArgSty{OR} \FuncArgSty{driving
  characteristical} \ArgSty{OR} \FuncArgSty{driving trait*}
\ArgSty{OR} \FuncArgSty{driving abilit*} \ArgSty{OR}
\FuncArgSty{driver* abilit*} \ArgSty{OR} \FuncArgSty{conductor*
  abilit*} )\;
\Indm
)\;
\ArgSty{OR}\;
  (\;
\Indp
\FuncSty{TI=}(\Indp\FuncArgSty{deep learning} \ArgSty{OR} \FuncArgSty{convolutional
  neural network} \ArgSty{OR} \FuncArgSty{convolution * network}
\ArgSty{OR} \FuncArgSty{deep neural network} \ArgSty{OR}
\FuncArgSty{auto encoder} \ArgSty{OR} \FuncArgSty{deep belief network}
\ArgSty{OR} \FuncArgSty{convolutional network} \ArgSty{OR}
\FuncArgSty{cnn} \ArgSty{OR} \FuncArgSty{dbn} \ArgSty{OR}
\FuncArgSty{deep architecture} \ArgSty{OR} \FuncArgSty{autoencoder}
\ArgSty{OR} \FuncArgSty{deep bayesian network} \ArgSty{OR}
\FuncArgSty{deep * network} \ArgSty{OR} \FuncArgSty{deep convolution
  network} \ArgSty{OR} \FuncArgSty{deep convolutional network}
\ArgSty{OR} \FuncArgSty{deep neural convolutional network} \ArgSty{OR}
\FuncArgSty{deep neural convolution network} \ArgSty{OR}
\FuncArgSty{deep autoencoder} \ArgSty{OR} \FuncArgSty{deep auto
  encoder} \ArgSty{OR} \FuncArgSty{deep * autoencoder} \ArgSty{OR}
\FuncArgSty{deep * auto encoder} \ArgSty{OR} \FuncArgSty{dnn}
\ArgSty{OR} \FuncArgSty{deep multilayer neural network} \ArgSty{OR}
\FuncArgSty{deep artificial neural network} \ArgSty{OR}
\FuncArgSty{deep boltzmann machine} \ArgSty{OR} \FuncArgSty{deep
  multitask learning} \ArgSty{OR} \FuncArgSty{deep extreme learning
  machine} \ArgSty{OR} \FuncArgSty{deep recurrent neural network}
\ArgSty{OR} \FuncArgSty{deep rnn} \ArgSty{OR} \FuncArgSty{deep fuzzy}
)\;
\ArgSty{AND}\;
\FuncSty{TI=}(\FuncArgSty{drive* identif*} \ArgSty{OR} \FuncArgSty{drive*
  fingerprint*} \ArgSty{OR} \FuncArgSty{drive* recogn*} \ArgSty{OR}
\FuncArgSty{drive* behaviour*} \ArgSty{OR} \FuncArgSty{drive*
  behavior*} \ArgSty{OR} \FuncArgSty{drive* model*} \ArgSty{OR}
\FuncArgSty{drive* perform*} \ArgSty{OR} \FuncArgSty{drive* modelling}
\ArgSty{OR} \FuncArgSty{drive* modeling} \ArgSty{OR}
\FuncArgSty{drive* feature} \ArgSty{OR} \FuncArgSty{drive* mapping}
\ArgSty{OR}  \FuncArgSty{drive* characteristic} \ArgSty{OR}
\FuncArgSty{drive* characteristical} \ArgSty{OR} \FuncArgSty{drive*
  trait*} \ArgSty{OR} \FuncArgSty{conductor* identif*} \ArgSty{OR}
\FuncArgSty{conductor* fingerprint*} \ArgSty{OR}
\FuncArgSty{conductor* recogn*} \ArgSty{OR} \FuncArgSty{conductor*
  behaviour*} \ArgSty{OR} \FuncArgSty{conductor* behavior*}
\ArgSty{OR} \FuncArgSty{conductor* model*} \ArgSty{OR}
\FuncArgSty{conductor* perform*} \ArgSty{OR} \FuncArgSty{conductor*
  modelling} \ArgSty{OR} \FuncArgSty{conductor* modeling} \ArgSty{OR}
\FuncArgSty{conductor* feature} \ArgSty{OR} \FuncArgSty{conductor*
  mapping} \ArgSty{OR} \FuncArgSty{conductor* characteristic}
\ArgSty{OR} \FuncArgSty{conductor* characteristical} \ArgSty{OR}
\FuncArgSty{conductor* trait*} \ArgSty{OR} \FuncArgSty{driving
  identif*} \ArgSty{OR} \FuncArgSty{driving fingerprint*} \ArgSty{OR}
\FuncArgSty{driving recogn*} \ArgSty{OR} \FuncArgSty{driving
  behaviour*} \ArgSty{OR} \FuncArgSty{driving behavior*} \ArgSty{OR}
\FuncArgSty{driving model*} \ArgSty{OR} \FuncArgSty{driving perform*}
\ArgSty{OR} \FuncArgSty{driving modelling} \ArgSty{OR}
\FuncArgSty{driving modeling} \ArgSty{OR} \FuncArgSty{driving feature}
\ArgSty{OR} \FuncArgSty{driving mapping} \ArgSty{OR}
\FuncArgSty{driving characteristic} \ArgSty{OR} \FuncArgSty{driving
  characteristical} \ArgSty{OR} \FuncArgSty{driving trait*}
\ArgSty{OR} \FuncArgSty{driving abilit*} \ArgSty{OR}
\FuncArgSty{driver* abilit*} \ArgSty{OR} \FuncArgSty{conductor*
  abilit*} )\;
\Indm)\;
\;
\FuncSty{Refined by: WEB OF SCIENCE CATEGORIES:} ( \FuncArgSty{COMPUTER SCIENCE ARTIFICIAL INTELLIGENCE} \ArgSty{OR} \FuncArgSty{ENGINEERING MULTIDISCIPLINARY} \ArgSty{OR} \FuncArgSty{COMPUTER SCIENCE CYBERNETICS} \ArgSty{OR} \FuncArgSty{COMPUTER SCIENCE INFORMATION SYSTEMS} \ArgSty{OR} \FuncArgSty{COMPUTER SCIENCE INTERDISCIPLINARY APPLICATIONS} \ArgSty{OR} \FuncArgSty{COMPUTER SCIENCE THEORY METHODS} \ArgSty{OR} \FuncArgSty{MATHEMATICS INTERDISCIPLINARY APPLICATIONS} \ArgSty{OR} \FuncArgSty{ENGINEERING ELECTRICAL ELECTRONIC} )\;
\FuncSty{Indexes=}\FuncArgSty{SCI-EXPANDED, SSCI, A\&HCI, CPCI-S, CPCI-SSH}, \FuncSty{ESCI Timespan=}\FuncArgSty{All years}
\end{algorithm*}

\subsection{Critérios de inclusão e exclusão}
\label{sec:protocolo:inclusao}
Os critérios de inclusão e exclusão se encarregaram da triagem dos artigos
que continham todas as premissas necessárias para se atingir o
objetivo desta pesquisa. Cada um dos critérios foi aplicado em um momento
distinto da revisão, os critérios de inclusão foram empregados na
atividade \emph{Segregação dos resultados}, pertencente à etapa do
levantamento bibliográfico, e os critérios de exclusão foram adotados
nas atividades de \emph{Filtragem pelo resumo} e \emph{Filtragem pela
  leitura}, concernente à etapa de seleção dos artigos.

\begin{enumerate}
\item Critérios de inclusão aplicados:
\begin{itemize}
\item Faz uso de técnicas de \textit{deep learning}
\item Realizada análise do comportamento do motorista
\item Utiliza dados provenientes de sensores
\end{itemize}
  \item Critérios de exclusão
\begin{itemize}
\item Escrito em outro idioma senão o inglês
\item Não descreve os procedimentos de análise comportamental
\item Dados sensoriados provenientes de celulares ou outros
  dispositivos não-embarcados
\item Não descreve os dados coletados do veículo
\end{itemize}
\end{enumerate}

\section{Resultados da pesquisa}
\label{sec:resultados}
Após a execução das \textit{strings} de busca em cada uma das bases de
pesquisa, foram apurados, ao todo, 172 artigos não únicos, 116
provenientes da plataforma Scopus e 56 da Web of Science. A seguir, os dois grupos de
artigos foram unificados em uma coleção de 124 artigos únicos. A base de pesquisa Scopus foi a que mais retornou artigos e também a
que mais retornou artigos únicos, são 67 contra 9 da Web of Science.

A imagem \ref{fig:distselecao} mostra o resultado da seleção dos
artigos únicos após a aplicação dos critérios de inclusão. A
verificação de aderência foi realizada em
conjunto com as atividades \emph{Filtragem pelo título} e
\emph{Filtragem pelo Resumo}, mesmo assim 15 artigos foram
classificados como \emph{incertos} pois nem o título nem o resumo dos textos eram
claros o suficiente para a tomada de decisão. Devido a quantidade de
artigos desconsiderados ser alta, os estudos classificados incertos foram selecionados para
leitura.

A média de artigos desconsiderados porque não atendiam completamente aos critérios de
inclusão é 62,33\%, sendo que a regra relacionada ao uso de dados
sensoriados automotivos possui a maior porcentagem, 42,61\% das
exclusões.

Ao todo, foram selecionados 25 artigos para análise crítica,
disponibilizados na tabela \ref{tab:artigosdossie} junto com a
identificação dos autores e ano de publicação. Esta tarefa tinha como
alvo, além de responder as questões de pesquisa já enunciada, entender
profundamente quais procedimentos precisam ser realizados para a
segmentação do comportamento do condutor, quais dados precisam coletados do
veículo para realizar tal segmentação, quais métodos de comparação foram
utilizados para avaliação e detecção de falsos-positivos e assimilar quais procedimentos devem ser utilizados no
tratamento dos dados coletados e como disponibilizá-los para a rede neural.

Ao término da tarefa acima, 15 pesquisas precisaram ser descartadas
pois não atendiam plenamente aos critérios de inclusão, a maioria
tratava-se de estudos rotulados como incertos. Os dois principais
fatores que levaram a exclusão foi a utilização de dispositivos
móveis, como celulares, GPS e outros dispositivos não embarcados, e o uso de imagens, capturadas em tempo real por câmeras
instaladas no veículo ou usando conjunto de dados públicos, para realizar a
atividade de análise comportamental do motorista. Os 8 artigos
restantes \cite{Lee2017}, \cite{Liang2014146}, \cite{Kagawa2017},
\cite{Liu2018}, \cite{Liu20151054}, \cite{Liu2016},
\cite{Liu20172477} e \cite{Liu20141427} cumprem
rigorosamente todos os critérios de inclusão e não se enquadram em
nenhum dos critérios de exclusão, por isso a discussão dos resultados,
apresentada na próxima sessão, será realizada baseando-se nos estudos
em questão.

\begin{figure}
\centering
\includegraphics[scale=0.25]{distribuicao_selecao}
\caption{Aderência dos artigos únicos aos critérios de inclusão. Os
  artigos que atendem a todos os critérios estão
  rotulados como aderentes, os que descumprem pelo menos
  uma das exigências são marcados como inaderentes.}
\label{fig:distselecao}
\end{figure}

\begin{landscape}
\begin{table}[!t]
\renewcommand{\arraystretch}{1.3}
\caption{Lista dos artigos que atendem aos critérios de inclusão e
  que fizeram parte da etapa \emph{Análise dos estudos selecionados}.}
\label{tab:artigosdossie}
\centering
\begin{tabular}{p{12cm}ll}
\multicolumn{1}{c}{\textbf{Artigo}}
  & \multicolumn{1}{c}{\textbf{Autores}}
  & \multicolumn{1}{c}{\textbf{Ano}} \\ \hline
A framework for evaluating aggressive driving behaviors based on in-vehicle driving records                                                      & Lee J., Jang K.                                                              & 2017                             \\
A hybrid Bayesian Network approach to detect driver cognitive distraction                                                                        & Liang Y., Lee J.D.                                                           & 2014                             \\
A neural network-based autonomous articulated vehicle system considering driver behavior                                                         & Zhang W.-M., Han H.-B., Yang J., Yi X.                                       & 2016                             \\
A rule-based neural network approach to model driver naturalistic behavior in traffic                                                            & Chong L., Abbas M.M., Medina Flintsch A., Higgs B.                           & 2013                             \\
Adding Intelligence to Cars Using the Neural Knowledge DNA                                                                                       & Zhang H., Li F., Wang J., Wang Z., Shi L., Zhao J., Sanín C., Szczerbicki E. & 2017                             \\
Analysis of driving skills based on deep learning using stacked autoencoders                                                                     & Kagawa T., Chandrasiri N.P.                                                  & 2017                             \\
Brain-inspired Cognitive Model with Attention for Self-Driving Cars                                                                              & Chen S., Zhang S., Shang J., Chen B., Zheng N.                               & 2017                             \\
Cognitive and personality determinants of fitness to drive                                                                                       & Sommer M., Herle M., Häusler J., Risser R., Schützhofer B., Chaloupka Ch.    & 2008                             \\
Context-aware driver behavior detection system in intelligent transportation systems                                                             & Al-Sultan S., Al-Bayatti A.H., Zedan H.                                      & 2013                             \\
DarNet: A deep learning solution for distracted driving detection                                                                                & Streiffer C., Raghavendra R., Benson T., Srivatsa M.                         & 2017                             \\
Deep Learning based Traffic Direction Sign Detection and Determining Driving Style                                                               & Karaduman M., Eren H.                                                        & 2017                             \\
DeepSafeDrive: A grammar-aware driver parsing approach to Driver Behavioral Situational Awareness (DB-SAW)                                       & Le T.H.N., Zhu C., Zheng Y., Luu K., Savvides M.                             & 2017                             \\
Defect-repairable latent feature extraction of driving behavior via a deep sparse autoencoder                                                    & Liu H., Taniguchi T., Takenaka K., Bando T.                                  & 2018                             \\
Design of a Speed Assistant to Minimize the Driver Stress                                                                                        & Liu H., Taniguchi T., Takenaka K., Bando T.                                  & 2018                             \\
Driver-Behavior Modeling Using On-Road Driving Data: A new application for behavior signal processing                                            & Miyajima C., Takeda K.                                                       & 2016                             \\
Essential feature extraction of driving behavior using a deep learning method                                                                    & Liu H., Taniguchi T., Tanaka Y., Takenaka K., Bando T.                       & 2015                             \\
Learning driver behavior models from traffic observations for decision making and planning                                                       & Gindele T., Brechtel S., Dillmann R.                                         & 2015                             \\
Multi-CNN and decision tree based driving behavior evaluation                                                                                    & Yin S., Duan J., Ouyang P., Liu L., Wei S.                                   & 2017                             \\
Nonintrusive detection of driver cognitive distraction in real time using Bayesian networks                                                      & Liang Y., Lee J.D., Reyes M.L.                                               & 2007                             \\
Reducing the negative effect of defective data on driving behavior segmentation via a deep sparse autoencoder                                    & Liu H., Taniguchi T., Takenaka K., Tanaka Y., Bando T.                       & 2016                             \\
Toward Intelligent Vehicle Intrusion Detection Using the Neural Knowledge DNA                                                                    & Li F., Zhang H., Wang J., Liu Y., Gao L., Xu X., Sanin C., Szczerbicki E.    & 2018                             \\
Toward Safer Highways: Predicting Driver Stress in Varying Conditions on Habitual Routes                                                         & Magana V.C., Munoz-Organero M.                                               & 2017                             \\
Towards neuroimaging real-time driving using convolutional neural networks: Development of CNNs for autonomous steering in the TORCS environment & Musoles C.F.                                                                 & 2017                             \\
Visualization of Driving Behavior Based on Hidden Feature Extraction by Using Deep Learning                                                      & Liu H., Taniguchi T., Tanaka Y., Takenaka K., Bando T.                       & 2017                             \\
Visualization of driving behavior using deep sparse autoencoder                                                                                  & Liu H., Taniguchi T., Takano T., Tanaka Y., Takenaka K., Bando T.            & 2014                            
\end{tabular}
\end{table}
\end{landscape}



\section{Discussão dos resultados}
\label{sec:discussao}
Esta sessão tem o objetivo de discorrer de forma crítica sobre os
artigos da tabela \ref{tab:artigoslimpos} e de responder as três questões
de pesquisa enunciadas na seção \ref{sec:protocolo:questoes}
baseando-se nos estudos e práticas dos artigos mencionados anteriormente.

A primeira questão de pesquisa está relacionada a identificação
das arquiteturas de redes neurais profundas que foram utilizadas no
processo de análise comportamental dos motoristas, e a partir das
leituras realizadas foi identificado que 7 estudos utilizam
\textit{deep auto encoder}, podendo variar em \textit{deep sparse auto
encoder} ou
\textit{stacked auto encoder}; o estudo restante aplicou
\textit{dynamic bayesian network}.

Os trabalhos mais completos foram
os desenvolvidos pelo Liu, H. \cite{Liu20151054}, \cite{Liu2016},
\cite{Liu20141427}, \cite{Liu20172477} e \cite{Liu2018} porque além de apresentarem
as fundamentações teóricas, utilizam técnicas como PCA, e algumas de
suas variações, para avaliar
dissimilaridades entre as aplicações dos algoritmos em diferentes
conjuntos de dados. As pesquisas sob sua liderança apresentam uma
maturidade acima das encontradas nos demais artigos, também quando
comparado ao grupo original de 25 artigos.

O artigo \textit{Defect-repairable latent feature extraction of driving behavior via a deep sparse autoencoder} \cite{Liu2018} é o único que aborda sistematicamente a
preocupação e a aplicação de informações contendo ruídos ou \textit{outliers}. O
estudo propõe o uso de um \textit{deep sparse auto encoder} (DSAE) treinado
com o objetivo de recuperar as características originais dos dados
corrompidos. A medição da qualidade é feita comparando seu desempenho
com outras implementações de DSAE, cuja diferença está na função de
ativação ou na maneira de propagação do erro através da rede neural
profunda, e com algumas implementações de PCA. Algo que foi
notado é o uso das mesmas técnicas já supracitadas ao efetuar as
comparações de qualidade.

No que diz respeito as características coletadas do veículo em tempo
real, foram apresentadas as seguintes:

\begin{itemize}
\item Taxa de abertura do acelerador
\item Pressão do cilindro mestre do freio
\item Ângulo de viragem
\item Velocidade das rodas
\item Leituras de medidor de velocidade
\item Velocidade do motor
\item Aceleração longitudinal
\item Aceleração lateral
\item Taxa de guinada
\end{itemize}
Nas pesquisas realizadas pela equipe do Liu, H., os
campos são combinados aleatoriamente, gerando pelo menos 8
\textit{datasets} diferentes para comparar o desempenho do
DSAE em cada uma das situações. Nas demais pesquisas não há justificativas para a ordem de
atributos utilizada no conjunto de dados.


\begin{table*}[!t]
\renewcommand{\arraystretch}{1.3}
\caption{Listagem dos artigos que não infringem nenhum dos critérios
  de inclusão e não se enquadram em nenhum dos critérios de exclusão.}
\label{tab:artigoslimpos}
\centering
\begin{tabular}{p{10cm}p{5cm}l}
\multicolumn{1}{c}{\textbf{Artigo}}                                                                           & \multicolumn{1}{c}{\textbf{Autores}}                              & \multicolumn{1}{c}{\textbf{Ano}} \\ \hline
A framework for evaluating aggressive driving behaviors based on in-vehicle driving records                   & Lee J., Jang K.                                                   & 2017                             \\
A hybrid Bayesian Network approach to detect driver cognitive distraction                                     & Liang Y., Lee J.D.                                                & 2014                             \\
Analysis of driving skills based on deep learning using stacked autoencoders                                  & Kagawa T., Chandrasiri N.P.                                       & 2017                             \\
Defect-repairable latent feature extraction of driving behavior via a deep sparse autoencoder                 & Liu H., Taniguchi T., Takenaka K., Bando T.                       & 2018                             \\
Essential feature extraction of driving behavior using a deep learning method                                 & Liu H., Taniguchi T., Tanaka Y., Takenaka K., Bando T.            & 2015                             \\
Reducing the negative effect of defective data on driving behavior segmentation via a deep sparse autoencoder & Liu H., Taniguchi T., Takenaka K., Tanaka Y., Bando T.            & 2016                             \\
Visualization of Driving Behavior Based on Hidden Feature Extraction by Using Deep Learning                   & Liu H., Taniguchi T., Tanaka Y., Takenaka K., Bando T.            & 2017                             \\
Visualization of driving behavior using deep sparse autoencoder                                               & Liu H., Taniguchi T., Takano T., Tanaka Y., Takenaka K., Bando T. & 2014                            
\end{tabular}
\end{table*}

\section{Conclusão e oportunidades}
\label{sec:conclusao}
Conforme é possível observar na imagem \ref{fig:evolcitacoes}, o
interesse pelos artigos utilizados nesta pesquisa têm atraído
crescente atenção da comunidade. A popularização das técnicas de deep
learning se deve principalmente aos avanços conquistados graças à sua
capacidade de aprender tarefas complexas. Entretanto, conforme as atividades desta
pesquisa avançavam foi perceptível a escassez de trabalhos que
abordassem a análise comportamental de motoristas conforme estruturado
na seção \ref{sec:protocolo}.

Uma parcela considerável dos artigos precisou ser desconsiderada por
utilizar celulares na captura de informações veiculares, e a
justificativa dada pelos autores, quando presente, recaía no preço para se adquirir
o equipamento embarcado de leitura. Também foram descartados artigos
que  analisavam características fisiológicas dos
motoristas como se fossem comportamentais.
Apesar dos contra-pontos citados acima, os trabalhos que atendiam as
necessidades impostas demonstravam com clareza e objetividade a forma
de condução da pesquisa, as técnicas aplicadas e os resultados
obtidos. Dentre estes artigos, aqueles produzidos pelo auto Liu,
H. possuíam maturidade acima dos outros. A pesquisa conduzida por ele
iniciou no ano de 2015 e vem evoluído deste então.

Ao final deste estudo é claro que as \textit{strings} de busca
precisam ser revisadas, tanto para facilitar sua leitura e
entendimento quanto para aumentar a quantidade de pesquisas devolvidas
pleas fontes de pesquisa. Também é possível explorar outras
plataformas de indexação de artigos com a finalidade de aumentar a
diversidade de artigos disponíveis para estudo.

\begin{figure*}[!h]
\centering
\includegraphics[scale=1]{evolucao_citacoes}
\caption{Evolução das citações dos 25 artigos selecionados para
  análise crítica}
\label{fig:evolcitacoes}
\end{figure*}

% An example of a floating figure using the graphicx package.
% Note that \label must occur AFTER (or within) \caption.
% For figures, \caption should occur after the \includegraphics.
% Note that IEEEtran v1.7 and later has special internal code that
% is designed to preserve the operation of \label within \caption
% even when the captionsoff option is in effect. However, because
% of issues like this, it may be the safest practice to put all your
% \label just after \caption rather than within \caption{}.
%
% Reminder: the "draftcls" or "draftclsnofoot", not "draft", class
% option should be used if it is desired that the figures are to be
% displayed while in draft mode.
%
%\begin{figure}[!t]
%\centering
%\includegraphics[width=2.5in]{myfigure}
% where an .eps filename suffix will be assumed under latex, 
% and a .pdf suffix will be assumed for pdflatex; or what has been declared
% via \DeclareGraphicsExtensions.
%\caption{Simulation results for the network.}
%\label{fig_sim}
%\end{figure}

% Note that the IEEE typically puts floats only at the top, even when this
% results in a large percentage of a column being occupied by floats.
% However, the Computer Society has been known to put floats at the bottom.


% An example of a double column floating figure using two subfigures.
% (The subfig.sty package must be loaded for this to work.)
% The subfigure \label commands are set within each subfloat command,
% and the \label for the overall figure must come after \caption.
% \hfil is used as a separator to get equal spacing.
% Watch out that the combined width of all the subfigures on a 
% line do not exceed the text width or a line break will occur.
%
%\begin{figure*}[!t]
%\centering
%\subfloat[Case I]{\includegraphics[width=2.5in]{box}%
%\label{fig_first_case}}
%\hfil
%\subfloat[Case II]{\includegraphics[width=2.5in]{box}%
%\label{fig_second_case}}
%\caption{Simulation results for the network.}
%\label{fig_sim}
%\end{figure*}
%
% Note that often IEEE papers with subfigures do not employ subfigure
% captions (using the optional argument to \subfloat[]), but instead will
% reference/describe all of them (a), (b), etc., within the main caption.
% Be aware that for subfig.sty to generate the (a), (b), etc., subfigure
% labels, the optional argument to \subfloat must be present. If a
% subcaption is not desired, just leave its contents blank,
% e.g., \subfloat[].


% An example of a floating table. Note that, for IEEE style tables, the
% \caption command should come BEFORE the table and, given that table
% captions serve much like titles, are usually capitalized except for words
% such as a, an, and, as, at, but, by, for, in, nor, of, on, or, the, to
% and up, which are usually not capitalized unless they are the first or
% last word of the caption. Table text will default to \footnotesize as
% the IEEE normally uses this smaller font for tables.
% The \label must come after \caption as always.
%
%\begin{table}[!t]
%% increase table row spacing, adjust to taste
%\renewcommand{\arraystretch}{1.3}
% if using array.sty, it might be a good idea to tweak the value of
% \extrarowheight as needed to properly center the text within the cells
%\caption{An Example of a Table}
%\label{table_example}
%\centering
%% Some packages, such as MDW tools, offer better commands for making tables
%% than the plain LaTeX2e tabular which is used here.
%\begin{tabular}{|c||c|}
%\hline
%One & Two\\
%\hline
%Three & Four\\
%\hline
%\end{tabular}
%\end{table}


% Note that the IEEE does not put floats in the very first column
% - or typically anywhere on the first page for that matter. Also,
% in-text middle ("here") positioning is typically not used, but it
% is allowed and encouraged for Computer Society conferences (but
% not Computer Society journals). Most IEEE journals/conferences use
% top floats exclusively. 
% Note that, LaTeX2e, unlike IEEE journals/conferences, places
% footnotes above bottom floats. This can be corrected via the
% \fnbelowfloat command of the stfloats package.




%\section{Conclusion}
%The conclusion goes here.





% if have a single appendix:
%\appendix[Proof of the Zonklar Equations]
% or
%\appendix  % for no appendix heading
% do not use \section anymore after \appendix, only \section*
% is possibly needed

% use appendices with more than one appendix
% then use \section to start each appendix
% you must declare a \section before using any
% \subsection or using \label (\appendices by itself
% starts a section numbered zero.)
%


% \appendices
% \section{Proof of the First Zonklar Equation}
% Appendix one text goes here.

% you can choose not to have a title for an appendix
% if you want by leaving the argument blank
% \section{}
% Appendix two text goes here.


% % use section* for acknowledgment
% \ifCLASSOPTIONcompsoc
%   % The Computer Society usually uses the plural form
%   \section*{Acknowledgments}
% \else
%   % regular IEEE prefers the singular form
%   \section*{Acknowledgment}
% \fi


%The authors would like to thank...


% Can use something like this to put references on a page
% by themselves when using endfloat and the captionsoff option.
\ifCLASSOPTIONcaptionsoff
  \newpage
\fi



% trigger a \newpage just before the given reference
% number - used to balance the columns on the last page
% adjust value as needed - may need to be readjusted if
% the document is modified later
\IEEEtriggeratref{8}
% The "triggered" command can be changed if desired:
%\IEEEtriggercmd{\enlargethispage{-5in}}

% references section

% can use a bibliography generated by BibTeX as a .bbl file
% BibTeX documentation can be easily obtained at:
% http://mirror.ctan.org/biblio/bibtex/contrib/doc/
% The IEEEtran BibTeX style support page is at:
% http://www.michaelshell.org/tex/ieeetran/bibtex/
\renewcommand\refname{Referências}
\bibliographystyle{IEEEtran}
% argument is your BibTeX string definitions and bibliography database(s)
\bibliography{IEEEabrv,bib/scopus}
%
% <OR> manually copy in the resultant .bbl file
% set second argument of \begin to the number of references
% (used to reserve space for the reference number labels box)
%\begin{thebibliography}{1}

% \bibitem{IEEEhowto:kopka}
% H.~Kopka and P.~W. Daly, \emph{A Guide to \LaTeX}, 3rd~ed.\hskip 1em plus
%   0.5em minus 0.4em\relax Harlow, England: Addison-Wesley, 1999.

%\end{thebibliography}

% biography section
% 
% If you have an EPS/PDF photo (graphicx package needed) extra braces are
% needed around the contents of the optional argument to biography to prevent
% the LaTeX parser from getting confused when it sees the complicated
% \includegraphics command within an optional argument. (You could create
% your own custom macro containing the \includegraphics command to make things
% simpler here.)
%\begin{IEEEbiography}[{\includegraphics[width=1in,height=1.25in,clip,keepaspectratio]{mshell}}]{Michael Shell}
% or if you just want to reserve a space for a photo:

% \begin{IEEEbiography}{Michael Shell}
% Biography text here.
% \end{IEEEbiography}

% % if you will not have a photo at all:
% \begin{IEEEbiographynophoto}{John Doe}
% Biography text here.
% \end{IEEEbiographynophoto}

% % insert where needed to balance the two columns on the last page with
% % biographies
% %\newpage

% \begin{IEEEbiographynophoto}{Jane Doe}
% Biography text here.
% \end{IEEEbiographynophoto}

% You can push biographies down or up by placing
% a \vfill before or after them. The appropriate
% use of \vfill depends on what kind of text is
% on the last page and whether or not the columns
% are being equalized.

%\vfill

% Can be used to pull up biographies so that the bottom of the last one
% is flush with the other column.
%\enlargethispage{-5in}



% that's all folks
\end{document}



%%% Local Variables:
%%% mode: latex
%%% TeX-master: t
%%% End:
